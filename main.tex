\documentclass{article}
\usepackage[utf8]{inputenc}
\usepackage{amsmath}
\usepackage{amssymb}
\usepackage{amsthm}
\usepackage{geometry}
\usepackage{showlabels}
\setlength{\textwidth}{426pt}
\usepackage{graphicx}
\graphicspath{ {./images/} }
\usepackage{amsmath,amssymb}
\usepackage{fullpage}
\DeclareMathOperator{\ran}{ran}
\DeclareMathOperator{\gra}{gra}
\usepackage{color}   %May be necessary if you want to color links
\usepackage{hyperref}
\hypersetup{
    colorlinks=true, %set true if you want colored links
    linktoc=all,     %set to all if you want both sections and subsections linked
    linkcolor=blue,  %choose some color if you want links to stand out
}

\newcommand{\red}{\color{red}}
\newcommand{\blue}{\color{blue}}
\newcommand{\black}{\color{black}}
\newcommand{\orange}{\color{orange}}

\title{Math 395}
\author{William Chang \footnote{chan087@usc.edu}}
\date{}
\begin{document}
\numberwithin{equation}{section}
\renewcommand{\baselinestretch}{1.5}
\setlength{\columnsep}{1in}
\maketitle
\tableofcontents
\newpage

\section{Problem Set 1}

\begin{enumerate}

\item Prove that $\sqrt{2}$ is irrational.

\begin{proof}
We prove by contradiction by assuming $\sqrt{2}$ is rational and can be written is as $\frac{q}{p}$ where $q$ and $p$ are relatively prime. It follows that $2p^2 = q^2$, and $q$ is an even number. By rearranging, we have $p^2 = 2*(\frac{q}{2})^2$ is also even \blue You shouldn't write it this way and instead use the equation $2p^2 = q^2$ to deduce that $2|q^2$. Since $2$ is prime, $2|q^2 \implies 2|q$ the rest follows as you have written it\black, implying that $q$ is also even, contradicting our assumption that $p$ and $q$ are relatively prime. Therefore, $\sqrt{2}$ is not rational.
\end{proof}

\item Suppose every point in $\mathbb{R}^3$ is coloured red, blue or green. Prove that for one
of these colours, every positive real number occurs as the distance between two points of this colour.

\begin{proof}

\end{proof}

\item Prove that there is no non-constant polynomial $p(x)$ with integer coefficients such
that $p(0), p(1), p(2),...$ are all prime numbers. [Harder but still do-able: Prove
that there is no non-constant polynomial $p(x)$ with integer coefficients such that
$p(1), p(2), p(3),...$ are all prime numbers.]

\begin{proof}

\end{proof}

\item Prove that any $n \in \mathbb{N}$ can be represented as $n = \pm 1^2 \pm 2^2 \pm 3^2 \pm \cdots \pm k^2$
for some $k \in \mathbb{N}$ and some choice of signs.

\begin{proof}
We first note the following identity which holds for all integer $k$: $(k+3)^2 - (k+2)^2 - (k+1)^2 + k^2 = 4$. Thus, if we have $n = \pm 1^2 \pm 2^2 \pm 3^2 \pm \cdots \pm K^2$, we can find a corresponding expression for $n+4$ by considering $n+4 = \pm 1^2 \pm 2^2 \pm 3^2 \pm \cdots \pm K^2 + k^2 - (k+2)^2 - (k+1)^2 + (k+3)^2$. Thus, if we can show that this statement holds for $n = 1, 2, 3, 4$ then we are done. This is completed with the following expressions
\begin{align}
    1 &= 1^2\\
    2 &= -1^2 - 2^2 - 3^2 + 4^2\\
    3 &= -1^2 + 2^2\\
    4 &= 1^2 - 2^2 - 3^2 + 4^2
\end{align}
\end{proof}

\item  Let $F_n$ be the Fibonacci numbers. Prove that $F_{2n+1} = F_{n+1}^2+ F_n^2$ for all $n \geq 0$. 
\begin{proof}

\end{proof}

\item Prove that for any $n$ distinct positive integers $a_1,..., a_n$,
\begin{equation}
    a_1^2 + a_2^2 + \cdots + a_n^2 \geq \frac{2n+1}{3}(a_1 + a_2 + \cdots + a_n)
\end{equation}

\begin{proof}
We prove this by induction. For $n=1$ the given inequality reduces to $a_1^2 \geq a_1$, which is trivially true for any positive integer $a_1$. For the inductive step we suppose that the statement is true for $n=k$; we now prove it for $n=k+1$. Let us suppose, WLOG, that $a_1 < a_2< \cdots < a_n$. Since each $a_i$'s are distinct integers, we have $a_n \geq n \implies a_n^2 \geq \frac{2n+1}{3}a_n$. Adding $\frac{2n+1}{3}(a_1 + \cdots a_n)$ to both sides of the inequality yields
\begin{equation}
   \frac{2n+1}{3}(a_1 + \cdots a_{n}) + a_{n+1}^2  \geq  \frac{2n+1}{3}(a_1 + \cdots a_{n+1})
\end{equation}

By the inductive hypothesis, the inequality above implies $a_1^2 + \cdots a_{n+1}^2 \geq \frac{2n+1}{3}(a_1 + \cdots a_{n+1})$ as desired. 
\end{proof}

\item  A chess player trains by playing at least one game per day, but no more than
$12$ games per week. Prove that there is a period of consecutive days in which he
plays exactly $20$ games.

\begin{proof}
Consider the quantity $G_n$ where $G_n$ is the total number of games played up to day $n$. We can reformulate this problem as finding $a, b$ such that $G_a - G_b = 20$. By the conditions given in the problem we have, 
\begin{align}
    G_{n+7} - G_n &\leq 12\\
    G_{n+1} - G_n &\geq 1
\end{align}

By convention, we define $G_0 = 0$. Now consider the increasing sequence $G_0, G_1,..,G_{20}$. Since $G_{n+1} - G_n \geq 1$ , it follows that this is an increasing sequence. Furthermore, by $ G_{n+7} - G_n \leq 12$, we have $G_{20} - G_0 \leq 36$. Considering each of the $G_i$'s in this sequence $\mod 20$, by pigeonhole principle, it follows that there exists $a > b$ such that $G_a \equiv G_b \mod 20$, which implies $G_a - G_b = 20k$ for some nonnegative integer $k$. if we can show that $k = 1$ then we are done. However, $G_a - G_b \leq G_{20} -G_0  \leq 36$ so $k \leq 1$. Furthermore, since $G_a - G_b > 0$, it follows that $k \geq 1$. Thus, $k = 1$ and we are done.  

\end{proof}

\end{enumerate}
\end{document}
